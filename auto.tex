\chapter{Применение изученных методов в системах с автопилотом}

В современном мире разработка автономного транспорта требует систем обработки изображения для обнаружения дорожных знаков. Такая область, как компьютерное зрение, является главным ведущим звеном в решении этой задачи.

 Рассмотренные в прошлых частях алгоритмы обработки снимков для последующего детектирования объектов широко применяются в этой сфере.
 %В зависимости от требований, может быть выбран определенный алгоритм, который показывает наилучший результат при имеющихся входных данных.
 
 Так, например, методы поиска контуров комбинируют с алгоритмами сравнения с эталоном. В связи с тем, что снимки с камер систем автопилота не всегда высокого качества и часто имеют шумы из-за плохой погоды или высокой скорости движения, сначала используют методы аналогичные детектору границ Кэнни для того, чтобы определить края дорожного знака (рисунок~\ref{img:kenny}~\cite{znak_cv}). Следующим этапом является сравнение участков полученного изображения с эталоном дорожного знака для распознания его на снимке. В случае совпадения определенных ранее границ с шаблоном, обнаруженный знак относят к классу эталона.
\imgScale{0.7}{kenny}{Определение краев дорожного знака}

Преобразование Хафа, относящееся к методам сравнения с эталоном, не редко используется как самостоятельный алгоритм распознавания дорожных знаков. Его суть заключается в поиске объектов на изображении, которые относятся к определенному классу фигур. Однако в случае устройств систем с автопилотом, когда на вход алгоритму подаются снимки, имеющие различные дефекты (шумы, засветы, размазанность картинки вследствии нестабилизированности камеры и т.~д.), этот метод показывает не лучшие результаты. Поэтому для исправления такой ситуации используют различные способы предварительной обработки полученных снимков~\cite{some_inf}.
 
Однако исходя из информации, полученной в ходе исследования, самым приоритетным вариантом решения задачи детектирования дорожных знаков на снимке являются сверточные нейронные сети (рисунок \ref{img:neural_net}~\cite{neural_struct}). Существует большое количество моделей таких сетей, например: \textit{STN} (spatial transformer network), \textit{IDSIA} и \textit{MultiNet}. Данный подход применяется во многих устройствах с автопилотом, так как такие системы более устойчивы к помехам на снимке, они обладают более высокой скоростью и точностью распознавания дорожных знаков вне зависимости от состояния знака или условий съемки. Нейронные сети позволяют получать намного больше информации об изображении, которая в дальнейшем может быть использована для <<тренировки>> модели. Главным преимуществом такого метода является возможность постоянного дообучения системы, что впоследствии приводит к увеличению качества классификации и обнаружения дорожных знаков.

\imgScale{0.9}{neural_net}{Пример структуры сверточной нейронной сети}
