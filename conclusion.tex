\chapter*{ЗАКЛЮЧЕНИЕ}
\addcontentsline{toc}{chapter}{ЗАКЛЮЧЕНИЕ}
Из приведенного анализа и сравнений становится ясно, что невозможно выделить единственный универсальный метод решения задачи коммивояжера, превосходящий все остальные по всем критериям. Точные методы гарантируют оптимальное решение, но становятся неприменимыми при большом числе городов из-за экспоненциальной сложности. В то же время эвристические алгоритмы работают гораздо быстрее, масштабируются на большие размеры задачи, однако не дают стопроцентной гарантии достижения оптимума.

Таким образом, выбор метода определяется спецификой поставленной задачи, ее размерностью, доступными вычислительными ресурсами и предъявляемыми требованиями к качеству решения.

% В реальных условиях зачастую прибегают к гибридным или эвристическим подходам, обеспечивающим разумный компромисс между временем решения и качеством результата.


Поставленная цель научно-исследовательской работы --- сравнительный анализ существующих методов решения задачи коммивояжера --- была успешно достигнута.

В ходе выполнения научно-исследовательской работы были решены следующие задачи:
\begin{itemize}[label=---]
    \item проведено исследование существующих методов решения задачи коммивояжера;
    \item определены преимущества и недостатки рассмотренных методов;
    \item cформулированы критерии сравнения методов;
    \item проведен сравнительный анализ методов.
\end{itemize}