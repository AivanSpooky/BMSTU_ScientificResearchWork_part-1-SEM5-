\addcontentsline{toc}{chapter}{СПИСОК ИСПОЛЬЗОВАННЫХ ИСТОЧНИКОВ}

\renewcommand\bibname{СПИСОК ИСПОЛЬЗОВАННЫХ ИСТОЧНИКОВ}
\begin{thebibliography}{}

\bibitem{1} Гэри М., Джонсон Д. Вычислительные машины и труднорешаемые задачи. --- М.: Мир, 1982. --- 419 с. 

\bibitem{2} В.М. Курейчик. Применение генетических алгоритмов для решения
комбинаторно-логических задач оптимизации. Интеллектуальные САПР.
Междуведомственный тематический научный сборник. Выпуск 5, Таганрог, 1995 г.

\bibitem{3} Боронихина Е. А. Точные и эвристические методы для решения задачи коммивояжера. – 2015 г.

\bibitem{4} КМеламед И. И., Сергеев С. И., Сигал И. Х. Задача коммивояжера. Точные методы //Автоматика и телемеханика. --- 1989 г.

\bibitem{5} Christofides N. The travelling salesman problem//Combinatorial Optimization. London, 1979. P. 131-149.

\bibitem{6} Беллман Р. Применение динамического программирования к задаче о коммивояжере // Кибернетический сб. Вып. 9. М.: Мир, 1964. С. 219-222.

\bibitem{7} Сергеев С. И. Вычислительные алгоритмы решения задачи коммивояжера I. Общая схема классификации //Автоматика и телемеханика. – 1994. – №. 5. – С. 66-79.

\bibitem{8} Henry-Labordere A. L. The record balancing problem: dynamic programming solution of a generalized travelling salesman problem//RIRO. 1969. 3An. B—2. P.
43-49. 

\bibitem{9} Saksena J. P., Kumar S. The routin problem with К specified nodes //Oper. Res.
1966. V. 14. P. 909-913.

\bibitem{10} Макаров И. П., Яворский В. В. Об одном обобщении задачи построения маршрута
коммивояжера//АиТ. 1975. № 4. С. 71-74. 


\bibitem{11} Коробков В. К., Кричевский Р. И. Некоторые алгоритмы для решения задачи
коммивояжера//Математические модели и методы оптимального управления.
Новосибирск: Наука, 1966. С. 106-108.

\bibitem{12} Калашникова Т. В. Исследование операций в экономике //Изд-во Томского Политехнического университета. – 2011г.

\bibitem{13} Шуть В. Н. и др. Два алгоритма приближённого решения задачи коммивояжёра. – 2002г.

\bibitem{14} Поборчий И. В. Исследование эвристических методов решения задачи коммивояжера. – 2016г.

\bibitem{15} Курейчик В. В., Курейчик В. М. Генетический алгоритм определения пути коммивояжера //Известия Российской академии наук. Теория и системы управления. – 2006г.

\bibitem{16} Гладков Л.А., Курейчик В.В., Курейчик В.М. // Генетические алгоритмы. М.:
Физматлит, 2006г.

\bibitem{17} Товстик Т. М., Жукова Е. В. Алгоритм приближенного решения задачи коммивояжера //Вестник Санкт-Петербургского университета. Математика. Механика. Астрономия. – 2013. – №. 1. – С. 101-109.

\end{thebibliography}


