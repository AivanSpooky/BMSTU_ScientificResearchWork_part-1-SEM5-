\chapter*{ВВЕДЕНИЕ}
\addcontentsline{toc}{chapter}{ВВЕДЕНИЕ}

Коммивояжер (фр. $commis voyageur$) — бродячий торговец. Задача коммивояжера $-$ важная задача транспортной логистики, отрасли, занимающейся планированием транспортных перевозок. Коммивояжеру, чтобы распродать нужные и не очень нужные в хозяйстве товары, следует объехать $n$ пунктов и в конце концов вернуться в исходный пункт. Требуется определить наиболее выгодный маршрут объезда. В качестве меры выгодности маршрута может служить суммарное время в пути, суммарная стоимость дороги, или, в простейшем случае, длина маршрута.

Задача коммивояжера является важной и вместе с тем трудноразрешимой [1]. Необходимость разработки эффективных методов ее решения обусловлена растущими объемами данных и сложностью их обработки. Логистические компании, службы доставки и транспортные системы ежедневно сталкиваются с необходимостью прокладки оптимальных маршрутов. Использование эффективных методов решения данной задачи позволяет экономить топливо, сокращать временные затраты и повышать качество сервиса, что важно при повышающейся конкуренции и стремлении к снижению расходов. То есть, задача коммивояжера актуальна как с научной, так и с практической точки зрения.

Цель научно-исследовательской работы $-$ сравнительный анализ существующих методов решения задачи коммивояжера.

Для достижения поставленной в работе цели предстоит решить следующие задачи:
\begin{itemize}[label=---]
    \item провести исследование существующих методов решения задачи коммивояжера;
    \item определить преимущества и недостатки рассмотренных методов;
    \item cформулировать критерии сравнения методов;
    \item провести сравнительный анализ методов.
\end{itemize}


