\chapter{Анализ предметной области}

\section{Основные определения}

Задача коммивояжера формулируется следующим образом: дано множество городов (вершин графа) и расстояний между каждым городом (весов ребер). Необходимо найти замкнутый маршрут, проходящий ровно по одному разу через каждый город и возвращающийся в исходную точку, при этом суммарная длина этого маршрута должна быть минимальной [2].

Граф --- абстрактная математическая структура, представляющая собой множество вершин (точек) и соединяющих их ребер (линий). Граф называют полным, если каждая пара вершин соединена ребром.

Гамильтонов путь --- путь в графе, проходящий через каждую вершину ровно один раз. Гамильтоновым циклом является такой цикл (замкнутый путь), который проходит через каждую вершину графа ровно по одному разу.
\imgScale{1}{imagesGam}{Пример Гамильтонова цикла}

Задача коммивояжера представляет собой задачу отыскания кратчайшего Гамильтонова цикла в полном конечном графе с $N$ вершинами. 

\section{Развитие методов решения}

Выделяют два типа решения этой задачи: точные и эвристические [3].

\subsection{Точные методы}
Точный метод --- алгоритмический подход, который гарантированно находит оптимальное решение задачи [4].

Для ЗК точные методы часто основаны на переборе всех перестановок городов (вершин) или применении методов динамического программирования, ветвей и границ или целочисленного программирования. Точные методы гарантируют нахождение оптимального решения, но их сложность возрастает экспоненциально с увеличением количества вершин.

\subsection{Эвристические методы}
Эвристический метод --- алгоритм, не гарантирующий нахождение строго оптимального решения, но в среднем быстро находящий «достаточно хорошие» решения.

Эвристические методы решения ЗК предназначены для поиска приближенного решения за приемлемое время, особенно для графов большой размерности.

\subsection{Гибридные методы}
Гибридные методы сочетают различные подходы (точные и эвристические) для достижения баланса между точностью, скоростью работы и вычислительными затратами. 


